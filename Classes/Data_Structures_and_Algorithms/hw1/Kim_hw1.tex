% Search for all the places that say "PUT SOMETHING HERE".

\documentclass[11pt]{article}
\usepackage{amsmath,textcomp,amssymb,geometry,graphicx,enumerate}

\def\Name{Josiah Kim}  % Your name
\def\SID{948500821}  % Your student ID number (e.g. 938231937)
\def\Login{juk483} % Your login (e.g. pzm11)
\def\Homework{1} % Number of Homework
\def\Session{Spring 2021}


\title{CMPSC 465 -- Spring 2021 --- Solutions to Homework \Homework}
\author{\Name, SID \SID, \texttt{\Login}}
\markboth{CMPSC 465 --\Session\  Homework \Homework\ \Name}{CMPSC 465 --\Session\ Homework \Homework\ \Name, \texttt{\Login}}
\pagestyle{myheadings}

\newenvironment{qparts}{\begin{enumerate}[{(}a{)}]}{\end{enumerate}}
\def\endproofmark{$\Box$}
\newenvironment{proof}{\par{\bf Proof}:}{\endproofmark\smallskip}

\textheight=9in
\textwidth=6.5in
\topmargin=-.75in
\oddsidemargin=0.25in
\evensidemargin=0.25in


\begin{document}
\maketitle

\section*{1. Acknowledgements}
\begin{qparts}
\item
I did not work in a group.
\item
I did not consult with any of my group members.
\item
I did not consult any non-class materials.
\end{qparts}


\newpage
\section*{2. Tribonaci numbers}
\begin{qparts}
\item
$R(i) = R(i-1) + R(i-2) + R(i-3)$ $\geq$ $3^{i/2}$ for all i $\geq$ 2 

BASE CASE: $i = 2$ \\
$R(2) = 3$ $\geq$ $3^{2/2}$ \\
$3 \geq 3$ \\
This is true.  

INDUCTIVE HYPOTHESIS: For $n > 2$, assume that the claim holds for all $2 \leq i \leq n$. \\
Assume $R(n)$ is true. \\
$R(n) = R(n-1) + R(n-2) + R(n-3)$ $\geq$ $3^{n/2}$

INDUCTIVE STEP: $i = n+1$ \\
$R(n+1) = R((n+1)-1) + R((n+1)-2) + R((n+1)-3)$ $\geq$ $3^{(n+1)/2}$ \\
$R(n+1) = R(n) + R(n-1) + R(n-2)$ $\geq$ $3^{(n+1)/2}$
$R(n+1) = R(n) + R(n-1) + R(n-2) \geq (R(n-1) + R(n-2) + R(n-3))*3^{1/2}$ (By I.H.)\\
$R(n+1) = {(R(n) + R(n-1) + R(n-2))}/{(R(n-1) + R(n-2) +R(n-3))} \geq 3^{1/2}$

This means $R(n) \geq 3^{n/2}$ is true for $n \geq 2$.\\
Therefore, $R(i) \geq 3^{n/2}$ is true for all $i \geq 2$.





\item
SEE BELOW
\end{qparts}


\newpage
\section*{3. Big Oh Definitions}
PROBLEM: Need to prove $O_{1}(g(n))$ $\subseteq $ $O_{2}(g(n))$ AND $O_{2}(g(n))$ $\subseteq $ $O_{1}(g(n))$. \\ 
\\
PROOF FOR $O_{1}(g(n))$ $\subseteq $ $O_{2}(g(n))$: 

Assume that $c_2 = 10$, $g(n) = n^2$. 

This means that $f(n) \leq 10n^2$. 

Assume that $c_1 = 10$, $g(n) = n^2$. 

This means that $f(n) \leq 10n^2$. 

Using the same constant and polynomial parameters, $f(n) \leq 10n^2$ in $O_{2}(g(n))$ is also part of $O_{1}(g(n))$. 

Therefore, $O_{1}(g(n))$ $\subseteq $ $O_{2}(g(n))$. \\
\\
PROOF FOR $O_{2}(g(n))$ $\subseteq $ $O_{1}(g(n))$. 

Assume that $c_1 = 25$, $g(n) = 3^n$. 

This means that $f(n) \leq 25(3^n)$. 

Assume $c_2 = 25$, $g(n) = 3^n$. 

This means that $f(n) \leq 25(3^n)$. 

Using the same constant and polynomial parameters, $f(n) \leq 25(3^n)$ in $O_{1}(g(n))$ is also part of $O_{2}(g(n))$. 

Therefore, $O_{1}(g(n))$ $\subseteq $ $O_{2}(g(n))$.\\
\\ 
CONCLUSION: 

It is true that $O_{1}(g(n))$ $\subseteq $ $O_{2}(g(n))$ AND $O_{2}(g(n))$ $\subseteq $ $O_{1}(g(n))$. 

Therefore, $O_{1}(g(n))$ $=$ $O_{2}(g(n))$ for all $g$.

\newpage
\section*{4. Analyze Running Time}
\begin{qparts}
\item
The "for" loop will have a time complexity of $n$.\\
The nested "while" loop will have a time complexity of $n$ as well. \\
We will represent the $j + 5$ as 1.\\
$T(n) = n * 1n$ \\
$T(n) = n^2$ \\
Therefore, $\Theta (n^2)$.


\item 
The "for" loop will have a time complexity of $n$. \\
The nested "for" loop will also have a time complexity of $n$. \\
The arithmetic step, $s + 2$ will be a constant, 1. \\
$T(n) = n * 1n$ \\
$T(n) = n^2$ \\
Therefore, $\Theta (n^2)$.


\item 
There are two simple arithmetic operations ($i \div 2$, $s + 1$) which will each have a time complexity of 1. \\
The "while" loop will have a time complexity of $n$.\\
The nested "for" loop will have a time complexity of $n$.\\
$T(n) = 1n * 1n$\\
$T(n) = n^2$\\
Therefore, $\Theta (n^2)$.


\item
The "for" loop will have a time complexity of $n$. \\
There are two operations which will each have a time complexity of 1. \\
The nested "while" loop will have a time complexity of $n$.\\
$T(n) = 1n * 1n$\\
$T(n) = n * n$\\
Therefore, $\Theta (n^2)$.

\end{qparts}





\newpage
\section*{5. Tighter Analysis}
Number, Flips

1, 1

2, 2

3, 1

4, 3

5, 1

6, 2

7, 1

8, 4

9, 1

10, 2

11, 1

12, 3

13, 1

14, 2

15, 1

16, 5

17, 1

18, 2

19, 1

20, 3

21, 1

22, 2

23, 1

24, 3

25, 1

26, 2

27, 1

28, 3

29, 1

30, 2

31, 1 \\

Total flips to 31: 56\\

Looking at the pattern of bit-flips, every four steps is about seven bit-flips. 

$T(n) = ((2^n -1)/4)*7 = \Theta (2^n)$

This statement holds true since $((2^n -1)/4)*7 \geq 2^n$.


\end{document}
