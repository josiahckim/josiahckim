% Search for all the places that say "PUT SOMETHING HERE".

\documentclass[11pt]{article}
\usepackage{amsmath,textcomp,amssymb,geometry,graphicx,enumerate,listings}

\def\Name{Josiah Kim}  % Your name
\def\SID{948500821}  % Your student ID number (e.g. 938231937)
\def\Login{juk483} % Your login (e.g. pzm11)
\def\Homework{3} % Number of Homework
\def\Session{Spring 2021}


\title{CMPSC 465 -- Spring 2021 --- Solutions to Homework \Homework}
\author{\Name, SID \SID, \texttt{\Login}}
\markboth{CMPSC 465 --\Session\  Homework \Homework\ \Name}{CMPSC 465 --\Session\ Homework \Homework\ \Name, \texttt{\Login}}
\pagestyle{myheadings}

\newenvironment{qparts}{\begin{enumerate}[{(}a{)}]}{\end{enumerate}}
\def\endproofmark{$\Box$}
\newenvironment{proof}{\par{\bf Proof}:}{\endproofmark\smallskip}

\textheight=9in
\textwidth=6.5in
\topmargin=-.75in
\oddsidemargin=0.25in
\evensidemargin=0.25in


\begin{document}
\maketitle

\section*{1. Acknowledgements}
\begin{qparts}
\item
I did not work in a group.
\item
I did not consult with any of my group members.
\item
I did not consult any non-class materials.
\end{qparts}

\newpage
\section*{2. Solving Recurrences}
\begin{qparts}
\item
1)$T(n) = 11T(\frac{n}{5}) + O(n^{1.3})$; $a = 11$, $b = 5$, $d = 1.3$ \\
$1.3$ ? $log_{5}11$ \\
$1.3 < 1.489896$ \\
$T(n) = O(n^{log_{5}n})$

2)$T(n) = 11T(\frac{n}{5}) + \Omega(n)$; $a = 11$, $b = 5$, $d = 1$ \\
$1$ ? $log_{5}11$ \\
$1 < 1.489896$ \\
$T(n) = \Omega(n^{log_{5}n})$ 

3)The answers agree. Therefore, $\Theta(n^{log_{5}n})$

\item
1)$T(n) = 6T(\frac{n}{2}) + O(n^{2.8})$; $a = 6$, $b = 2$, $d = 2.8$ \\
$2.8$ ? $log_{2}6$ \\
$2.8 > 2.58$ \\
$T(n) = O(n^{2.8})$

2)$T(n) = 6T(\frac{n}{2}) + \Omega(n)$; $a = 6$, $b = 2$, $d = 1$ \\
$1$ ? $log_{2}6$ \\
$1 < 2.58$ \\
$T(n) = \Omega(n^{log_{2}n})$ 

3)The answers do not agree. Therefore, we can use the upper bound as both the lower and upper bound: $\Theta(n^{2.8})$

\item
RECURSION TREE:\\
Branching Factor: 5 \\
Height: $log_{3}n$\\
Size: $\frac{n}{3^{k}}$\\
Number: $5^{k}$\\
$W_k$: $5^{k}log^{2}(\frac{n}{3^{k}})$\\
Total Work: $\sum_{k=0}^{log_{3}n} 5^{k}log^{2}(\frac{n}{3^{k}}) = \Theta(logn)$\\


\item
FOLDING METHOD:\\
$T(n) = T(n-2) + O(logn)$ \\
$T(n) = T(n-4) + O(logn) + O(logn)$\\
$T(n) = T(n-6) + O(logn) + O(logn) + O(logn)$ \\
$T(n) = T(n-2k) + kO(logn)$ \\
$\Theta(logn)$

\end{qparts}


\newpage
\section*{3.Sorted Array}
\begin{lstlisting}
def index_match(A, i=0):
	if i == A[i]:
		return i
	if i < A[n/2]:
		index_match(A[1...n/2], i++)
	if i > A[n/2]:
		index_match(A[n/2...n], i+(n/2))
	else:
		return false
\end{lstlisting}

This function takes two inputs: a list (A) and an index (i). If the index is equal to the value of the integer at the spefic index ($A[i] = i$), then it returns the index/integer. 

If not, then it checks if the index is less than or greater than the integer at the median ($A[n/2]$). If the integer at the median is less than the index, it will recursively call the function using the first half of the array and adding 1 to the index. Similarly, if the integer at the median is greater than the index, it will recursively call the function using the second half of the array and adding 1 to the index.

Since this algorithm does not does not iterate over every item in the array it would be quicker than $O(n)$. Furthermore, since the input is partitioned in half for every recursive call, this algorithm has a time complexity of $O(logn)$. If the input size were doubled, the algorithm would take one more step. 


\newpage
\section*{4.Linear Time Sorting}

\begin{lstlisting}

sort(unsorted):
	min = 0
	max = 0
	sorted = []
	counter = {}
	for x in unsorted:
		if x > max:
			max = x
		if x < min:
			min = x
	for y in range(min, max):
		counter[y] = 0
	for i in counter:
		for j in unsorted:
			if i == j:
				counter[j] += 1
	for z in counter:
		if counter[z] != 0:
			for w in range(counter[z]):
				sorted.append(z)
	return sorted




\end{lstlisting}

This function takes the argument of an unsorted list. The first loop iterates through the unsorted list and finds the min and max values which will have a time complexity of $O(n)$. 

The second "for" loop, initializes a dictionary, counter, with the key being the integers from min to max and the values being 0. This will result in the time complexity of $O(M)$. 

The third loop iterates over the counter again with its nested loop iterating over the unsorted list for each key of the dictionary. Once there is a match with the key of the dictionary and the integer in the unsorted list, the value is incremented by one. This section will result in the time complexity of $O(Mn)$. 
The final loop will then iterate over the counter. If the value of a key is not zero, it will append the key to the resulting list called sorted. Since this loop iterates over counter, it will result in another time complexity of $O(M)$.

Finally, the function returns the sorted list. We can see that the time complexity is dependent on M for three out of the four loops. Therefore, without knowing the relationship between n and M, it is safe to say that $O(n+M)$.


\newpage
\end{document}




















