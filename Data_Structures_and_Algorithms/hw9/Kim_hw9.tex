% Search for all the places that say "PUT SOMETHING HERE".

\documentclass[11pt]{article}
\usepackage{amsmath,textcomp,amssymb,geometry,graphicx,enumerate,listings,graphicx}

\def\Name{Josiah Kim}  % Your name
\def\SID{948500821}  % Your student ID number (e.g. 938231937)
\def\Login{juk483} % Your login (e.g. pzm11)
\def\Homework{9} % Number of Homework
\def\Session{Spring 2021}


\title{CMPSC 465 -- Spring 2021 --- Solutions to Homework \Homework}
\author{\Name, SID \SID, \texttt{\Login}}
\markboth{CMPSC 465 --\Session\  Homework \Homework\ \Name}{CMPSC 465 --\Session\ Homework \Homework\ \Name, \texttt{\Login}}
\pagestyle{myheadings}

\newenvironment{qparts}{\begin{enumerate}[{(}a{)}]}{\end{enumerate}}
\def\endproofmark{$\Box$}
\newenvironment{proof}{\par{\bf Proof}:}{\endproofmark\smallskip}

\textheight=9in
\textwidth=6.5in
\topmargin=-.75in
\oddsidemargin=0.25in
\evensidemargin=0.25in


\begin{document}
\maketitle


\section*{1. Getting started}
\begin{qparts}
\item
I did not work in a group.
\item
I did not consult with any of my group members.
\item
I did not consult any non-class materials.
\end{qparts}



\newpage
\section*{2. Heaviest Edge in a Cycle}

Let $A, B, C$ denote vertices of an undirected cycle in a graph $G$. The edge weights are as follows: $w_{(A, B)} = e* - 2$, $w_{(B, C)} = e* - 1$, $w_{(C, A)} = e*$. In this case, the edge connecting $C$ and $A$is the heaviest edge. We know that a MST does not contain any cycles and that at least two of these edges must be in the MST since an MST contains $|V|-1$ edges. \\

The MST can contain the conbination of these edges:\\
$w_{(C, A)} = e*$, $w_{(B, C)} = e* - 1$ \\
Total Cost: $2e*-1$\\
$w_{(C, A)} = e*$, $w_{(A, B)} = e* - 2$ \\
Total Cost: $2e*-2$\\
$w_{(A, B)} = e* - 2$, $w_{(B, C)} = e* - 1$ \\
Total Cost: $2e*-3$\\
 
 In the cases above, all the MST containing edge $(C, A)$ has a lighter MST that does not have edge $(C, A)$. Therefore, $e*$ cannot appear in any MST of G. 


\newpage
\section*{3. Huffman Encoding}
\begin{qparts}

\item 

Possible frequencies that would yield the following code would be: $\{f_a, f_b, f_b + f_c, f_c\}$. 


\item 

This code cannot be obtained because 0 is the prefix 00 which means that $a$ is in the path of $c$ making it not a full binary tree. 


\item 

This code cannot be obtained because it will not produce a full binary tree. The node connecting to the right of the root node will only have one child, $a$.


\end{qparts}

\newpage
\section*{4. Cost of a Prefix-Free Encoding}

\begin{lstlisting}
def find_encoding(f): 
	T = initialize empty tree
	H = priority queue ordered by f 
	W = priority queue ordered by c
	for i = 1 to n
    	insert(H, i)
    	insert(W, i)
	for k = n+1 to 2n-1
	    i = extractmin(H)
	    j = extractmin(H)
	    n = extractmin(W)
	    if f[i]+f[j] < f[i]+f[n]:
		    create node k in T with children i and j
		    f[k] = f[i] + f[j]
		    insert(H, k)
	    if f[i]+f[j] > f[i]+f[n]:
		    create node k in T with children i and n
		    f[k] = f[i] + f[n]
		    insert(H, k)

\end{lstlisting}
By adding another extractmin() to the huffman encoding we are adding another $O(logm)$. Therefore, the new total running time will be: $O(m(logm)^2)$.


\newpage
\end{document}












