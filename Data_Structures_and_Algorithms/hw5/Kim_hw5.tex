% Search for all the places that say "PUT SOMETHING HERE".

\documentclass[11pt]{article}
\usepackage{amsmath,textcomp,amssymb,geometry,graphicx,enumerate,listings}

\def\Name{Josiah Kim}  % Your name
\def\SID{948500821}  % Your student ID number (e.g. 938231937)
\def\Login{juk483} % Your login (e.g. pzm11)
\def\Homework{5} % Number of Homework
\def\Session{Spring 2021}


\title{CMPSC 465 -- Spring 2021 --- Solutions to Homework \Homework}
\author{\Name, SID \SID, \texttt{\Login}}
\markboth{CMPSC 465 --\Session\  Homework \Homework\ \Name}{CMPSC 465 --\Session\ Homework \Homework\ \Name, \texttt{\Login}}
\pagestyle{myheadings}

\newenvironment{qparts}{\begin{enumerate}[{(}a{)}]}{\end{enumerate}}
\def\endproofmark{$\Box$}
\newenvironment{proof}{\par{\bf Proof}:}{\endproofmark\smallskip}

\textheight=9in
\textwidth=6.5in
\topmargin=-.75in
\oddsidemargin=0.25in
\evensidemargin=0.25in


\begin{document}
\maketitle


\section*{1. Getting started}
\begin{qparts}
\item
I did not work in a group.
\item
I did not consult with any of my group members.
\item
I did not consult any non-class materials.
\end{qparts}



\newpage
\section*{2. DFS Basics}
\begin{qparts}
\item
A, B, D, E, G, F, C, H, I

\item
A(1, 12), B(2, 11), D(3 ,6), E(4, 5), G(7, 10), F(8, 9), C(13, 18), H(14, 17), I(15, 16)

\item
$A \rightarrow B$ = Tree

$B \rightarrow D$ = Tree 

$D \rightarrow E$ = Tree 

$E \rightarrow D$ = Back

$A \rightarrow E$ = Forward

$B \rightarrow G$ = Tree 

$G \rightarrow F$ = Tree 

$G \rightarrow D$ = Cross 

$B \rightarrow D$ = Tree 

$C \rightarrow H$ = Tree 

$H \rightarrow I$ = Tree 

$C \rightarrow I$ = Forward

\end{qparts}


\newpage
\section*{3.Pre and Post Processing}
\begin{qparts}

\item 
CLAIM: \\
if ${u, v}$ is an edge in an undirected graph, and during depthfirst search $post(u) < post(v)$, then $v$ is an ancestor of $u$ in the DFS tree.

PROOF: \\
There are only two cases in which $post(u) < post(v)$. Using nested interval notation we can depict the relationship between vertices.\\
Case 1: $[pre(u), post(u)][pre(v),post(v)]$ \\
Case 2: $[pre(v),[pre(u), post(u)], post(v)]$\\
Both casese represent a descendant-ancestor relationship. The first case is possible only if every other vertex has been visited and there is no edge between $u$ and $v$. However, we know that there is an edge between those two vertices. Therefore, only the second case is possible which means that $v$ is an ancestor of $u$.


\item 
We know that for $u$ to be an ancestor of $v$, $v$ would have to be the descendant of $u$. This is the case where $u$ is discovered prior to $v$ which means $[pre(u), [pre(v),post(v)], post(u)]$. More importantly, $post(u) > post(v)$.

Therefore, in $T$, we would get the pre and post processing numbers of node $v$ and compare it with node $u$'s pre and post processing numbers to see if $pre(u) < pre(v) < post(v) < post(u)$ holds true. If so, then $u$ is an ancestor of $v$. Else, it is not. 

This preprocessing would take constant time because the algorithm would always rely on the comparison of two nodes no matter the number of nodes in a path. 


\end{qparts}

\newpage
\section*{4.Application of DFS}

Let $G^R$ be the reverse graph of $G$.
\begin{lstlisting}

function smallest_j(G):
	while there are vertices unvisited in G:
		run DFS on G^R starting with the unvisited vertex 
			with the smallest i 
		for every vertex visited:
			m(j) = i
\end{lstlisting}
The values $m(1),..., m(n)$ can be computed in $O(|V|+|E|)$ time. We know this because we know that the time complexity for DFS is $O(|V|+|E|)$ for any graph. We are essentially just running a DFS on the reverse graph.  

\newpage
\end{document}
